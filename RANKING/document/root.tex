\documentclass[11pt,a4paper]{article}
\usepackage[T1]{fontenc}
\usepackage{isabelle,isabellesym}

\usepackage{amssymb}
  %for \<leadsto>, \<box>, \<diamond>, \<sqsupset>, \<mho>, \<Join>,
  %\<lhd>, \<lesssim>, \<greatersim>, \<lessapprox>, \<greaterapprox>,
  %\<triangleq>, \<yen>, \<lozenge>
\usepackage{amsmath}

\usepackage[ruled]{algorithm2e}

\SetKwInput{Init}{Initialization}
\SetKwInput{Online}{Online Matching}
\SetKwFor{Arrival}{On arrival of}{}{}
\SetKwIF{If}{ElseIf}{Else}{if}{}{else if}{else}{}

% this should be the last package used
\usepackage{pdfsetup}

% urls in roman style, theory text in math-similar italics
\urlstyle{rm}
\isabellestyle{it}

\begin{document}

\title{Formal Verification of the RANKING Algorithm for Online Bipartite Matching}
\author{Christoph Madlener}
\maketitle

\begin{abstract}
We provide a formalization of the RANKING algorithm as introduced by Karp, Vazirani, and
Vazirani~\cite{karp1990}, following a simplified proof by Birnbaum and Mathieu~\cite{birnbaum2008}.
We formalize that the competitive ratio is determined by instances
where a perfect matching exists. Based on this we prove the competitive ratio of $1 - \frac{1}{e}$
(in the limit) for general instances. We also give the first proof that the competitive ratio
of RANKING is determined by graphs with a perfect matching.
\end{abstract}

\tableofcontents

% sane default for proof documents
\parindent 0pt\parskip 0.5ex

\section{Introduction}
In the online bipartite matching problem we are given a bipartite graph $G=(U,V,E)$. The \emph{online}
vertices in $U$ arrive one by one, revealing the edges incident to them in $G$ on arrival. On
arrival of a vertex $u \in U$ we have to make the irrevokable decision to match it to one of
its \emph{offline} neighbors in $V$ (or not match it at all).

Karp et al.\ give a simple randomized algorithm, and prove its optimality~\cite{karp1990}. The
algorithm chooses a random \emph{ranking} (think total order) of the offline vertices $V$ and
matches any arriving $u \in U$ to the neighbor with the lowest rank (if possible):

\begin{algorithm}[H]
\DontPrintSemicolon
\caption{RANKING}\label{alg:ranking}
\Init{Choose a random permutation (ranking) $\sigma$ of $V$}
\Online{}
\Arrival{$u \in U$}{
  $N(u) \gets \text{set of unmatched neighbors of }u$\\
  \If{$N(u) \neq \emptyset$}{
    match $u$ to the vertex $v \in N(u)$ that minimizes $\sigma(v)$
  }
}
\end{algorithm}

Optimal in this case means that the algorithm reaches a competitive ratio of $1 - \frac{1}{e}$,
the best possible for online bipartite matching~\cite{karp1990}.
We give a formal proof that RANKING reaches a competitive ratio of $1-\frac{1}{e}$,
 following a proof due to Birnbaum and Mathieu~\cite{birnbaum2008}.

\hyperref[sec:prelims]{Section~\ref*{sec:prelims}} introduces necessary definitions and auxiliary lemmas for the rest of the
proof. In~\autoref{sec:ranking} we consider RANKING in a deterministic setting, i.e.\ we fix
an arbitrary ranking for the offline vertices. This section also includes the most involved part
of the formalization; proving for the first time that the competitve ratio is determined by instances where a perfect
matching exists in $G$. Finally, \autoref{sec:prob} states RANKING as a randomized algorithm and
deals with the probabilistic parts of the proof, yielding the competitive ratio in the end.

\section{Preliminaries}\label{sec:prelims}
We start by providing auxiliary material for relevant concepts used in the formalization.
In particular we need additional theory for lists, the model for permutations we 
chose~(\autoref{sec:more-list}).
In~\autoref{sec:more-graph} we define -- and prove properties of -- relevant graph theory.
Finally, in~\autoref{sec:misc} auxiliary lemmas for the probabilistic part of the formalization
are provided.

% generated text of all theories
\input{session}

% optional bibliography
\bibliographystyle{abbrv}
\bibliography{root}

\end{document}

%%% Local Variables:
%%% mode: latex
%%% TeX-master: t
%%% End:
